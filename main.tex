\documentclass[12pt,a4paper]{article}
\usepackage[margin=1in]{geometry}
\usepackage{booktabs}
\usepackage{setspace}
\usepackage{hyperref}


\onehalfspacing
\setcounter{secnumdepth}{4}

%=== COVER PAGE ===
\begin{document}
\begin{titlepage}
    \centering
    {\Large\bfseries CM10025 Programming 2: Personal Informatics System Report\par}
    \vspace{1cm}
    Group Name: Agile Analysts \\
    Group Number: 18 \\
    Date: March 24, 2025

    \vspace{2cm}
    \begin{tabular}{llll}
        \toprule
        Group Member & Username & Degree & Course \\
        \midrule
        David Cai & yc2800 & MComp Computer Science and Mathematics & Year 1 \\
        Mandeep Thakur & mt2434 &  MComp Computer Science and Mathematics & Year 1\\
        Jack Bancroft & jgb64 & BSc Computer Science & Year 1\\
        James Beagrie & jb4106 & BSc Computer Science and Mathematics & Year 1\\
        Satima Dosso & sd2745 & BSc Computer Science and Mathematics & Year 1 \\
    \end{tabular}
    \thispagestyle{empty}
\end{titlepage}

%=== TITLE + ABSTRACT (1 page) ===
\begin{center}
    {\Large\bfseries Personal Informatics System Report}\\[1ex]
    Author(s): David Cai, Mandeep Thakur, Jack Bancroft, James Beagrie, Satima Dosso
\end{center}
\begin{abstract}
    200 words summarising problem, solution, 3 sprint outcomes, next steps
    
\end{abstract}
\newpage

%=== TABLE OF CONTENTS ===
\tableofcontents
\newpage



%=== MAIN BODY (max 20 pages) ===

\section{Introduction (2 pages)}
Author: Mandeep Thakur
\subsection{Background Of The Problem}
University students often struggle with effective time management, leading to procrastination and poor academic performance. There are several studies that show a direct correlation between time management and academic achievement of university students [1]. First-year students, in particular, face the greatest amount of challenges [2]; getting familiar with a new environment and balancing both social and academic responsibilities along with the rapid rise of digital distractions make it increasingly difficult for students to allocate their time efficiently. This situation has created the need for a Personal Informatics (PI) system to offer insights into time usage habits and track daily routines. Personal Informatics (PI) is a field focused on assisting people by gathering, analysing and reflecting on personal data in order to better understand their habits and behaviours. These systems have emerged and convert raw data into actionable insights, enabling individuals to improve their behaviour by making their routines visible and quantifiable [3]. There is strong evidence that supports that PI systems affect user behaviour. According to this research [4], 38 \% of studies stated that PI systems led to noticeable changes in behaviour, which demonstrates how these systems improve users' expectations for their own outcomes while raising awareness of their own habits.

\subsection{Effectiveness of PI software systems}

User involvement, data accuracy, and the capacity to convey information in an engaging and meaningful manner are some aspects that greatly affect how effective these systems are. In order for a PI system to be effective, there must be several core principles the system must follow. Firstly, it must provide meaningful data visualisation. This enables students to quickly interpret patterns in their daily activities. Graphical depictions of the amount of time spent on various tasks can draw attention to periods of inefficiency and encourage users to review their schedules. Users are more likely to consider their habits and take action to change them when information is easily available and visually intuitive. Another core principle is that it must maintain long-term user motivation over time; a successful PI system must have habit-forming mechanisms. Streaks and progress awards are examples of features that can motivate users to engage with the system on a regular basis. According to behavioural psychology studies, users are more likely to maintain a habit when they receive some type of reward for regular participation. [Citation]. Furthermore, including reminders at certain times can help support positive habits by reminding the user to interact with the system, which not only helps the user stay accountable but also helps gather more data to analyse their actions over time and provide more accurate reports on their habits. One of the main problems with PI systems is the accuracy of data input as well as the method of collecting data from the user. A system that requires time-consuming manual procedures to enter data may discourage users from using the system on a regular basis.  In order to avoid this, automated data tracking can be implemented in several ways, such as using calendar applications and sensors. When manual input is necessary, the system should avoid making it unnecessarily complicated and time-consuming, for example, by introducing quick-entry options such as voice commands or predictive text.

\subsection{Introducing our PI system}

\label{sec:intro}


\section{Agile Software Process Planning and Management (2 pages)}
Authors:
\label{sec:agile}


\section{Software Requirements Specification (5 pages)}
Authors:
\label{sec:requirements}
\subsection{Requirements Gathering}

\subsection{Use Cases}

\subsection{Functional Requirements}

\subsection{Non‑Functional Requirements}


\section{Design (5 pages)}
Authors:
\label{sec:design}
\subsection{UML Class Diagrams}

\subsection{Sequence Diagrams}

\subsection{Design Rationale}


\section{Software Testing (Verification) (2 pages)}
Authors:
\label{sec:testing}
\subsection{Test Plans}

\subsection{Test Results}


\section{Reflection and Conclusion (4 pages)}
Authors:
\label{sec:reflection}
\subsection{System Critique}

\subsection{Process Critique}


%=== REFERENCES (does not count towards page limit) ===
\newpage
\bibliographystyle{ieeetr}
\section*{References}
% \bibliography{references}

%=== APPENDICES ===
\newpage
\appendix

\section{Group Contribution Form}
% Insert completed GCF

\section{Meeting Minutes}
% Attach meeting minutes

\section{Interview Transcripts}
% Include transcripts

\end{document}
